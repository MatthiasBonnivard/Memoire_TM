\selectlanguage{english}

The goal of this work is the creation of a neurorehabilitation serious game (SG) for the lower limbs using the rehabilitation device called Lambda Health System (LHS). Users are suffering of paresis in those limbs and have to learn to walk again. Those people may also suffer of cognitive troubles like hemineglect which is an important concept to keep in mind for the realization of the SG.
\\

The SG main goal is to add ludic aspects to physiotherapist's exercise which consists of the repetitive walking movements. The three other important aspects of this SG: high immersion; environment able to be impoverished according to the patient's cognitive abilities; visualization of the walking movement. The later aims to enabling mirror neurons and to enhancing the rehabilitation process. 
\\
 
The developped SG is an exploration game where the user, guided on a path, has to walk to be able to progress with the trigger to find civilisation's vestiges. The distribution of their left/right placement is given by the therapist.
\\

The look is retrieved with an Head Mounted Display. Such device is well known to be one of the best nowadays for visual immersion. Furthermore, the hearing sense is stimulated by footsteps noises, background sound, music and UI sounds. The only sense awaited but not actually present is the touch. He is supposed to be stimulated by haptic rendering in the LHS. Relating to the impoverishment, nine factors are implemented. They change the geometries, the light effects, the animations, the sounds, the colors and the density of the world. For the movement representation, the game is in first person and the avatar's legs are animated as the patient's ones. Another virtual character is ahead of the user all along the path doing a walking animation.
\\

Due to current official testings in the "Centre Hospitalier Universitaire Vaudois" (CHUV), the integration of the SG in the LHS system is prepared but not tested yet. The walking movement is simulated by a keyboard key. However, a meeting with a medical doctor was scheduled to show the actual SG. Encouraging feedback was received regarding its utility even for patients suffering of cognitive troubles, without a movement therapy.
\\

Keywords: serious game, lower limb neurorehabilitation, virtual reality, virtual world, immersion, cognitive impoverishment, hemineglect, multimodal.

\selectlanguage{french}