Ce projet est réalisé en tant que travail de Master pour conclure la formation "Master of Science in Engineering" en "Technologies de l'Information et de la Communication" proposée par la HES-SO. Il consiste à la conception et au développement d'un \textit{serious game} (SG) de marche avec une grande immersion pour la réhabilitation des jambes chez des patients souffrant de traumatismes neurologiques. Ce travail est contenu et est réalisé en amont d'un projet de recherche et développement (projet R\&D) dont il partage une partie des objectifs. Il fait également suite à plusieurs projets, dont le plus récent est "\textit{Serious Games for Rehabilitation}" (SG4R). Ce dernier fut réalisé en collaboration entre la HE-Arc, l'hepia, et la HEIG-VD ainsi qu'en étroite collaboration avec le CHUV et la Haute Ecole de Santé de Lausanne (HESAV). Il a donné lieu à une plateforme de prototypes de SGs fonctionnels, qui ont fait l'objet d'un user test auprès de spécialistes de la santé au CHUV.
\\

Un SG est un jeu où l'amusement est utilisé pour une intention sérieuse étant ici la réhabilitation des jambes. Le but est que le patient (premier utilisateur) fasse sa réhabilitation tout en s'amusant. Il faut également veiller à éviter d'ajouter de la complexité pour le thérapeute (second utilisateur) qui est habitué à travailler avec des exercices et non des jeux. Les patients sont des personnes devant réapprendre un mouvement de marche. La réhabilitation ciblée n'est pas musculaire mais neurologique (neuroréhabilitation) et se base sur la capacité de notre cerveau à refaire de nouvelles connexions (neuroplasticité). Son objectif est de récréer ou de remodeler les connexions inter-neuronales perdues à la suite d’un accident. Les exercices de rééducation consistent à réaliser des mouvements répétitifs pour que le cerveau réapprenne à les faire. %TODO Trouver une description plus précise des utilisateurs
\\

Les projets antérieurs ont également permis l'établissement d'une communication et le début d'une intégration entre les jeux réalisés à l'aide du logiciel \textit{Unity} et le dispositif de réhabilitation "Lambda". Le dispositif est souvent appelé "LHS" -- de "Lambda Health System" -- qui est en réalité le nom de la start-up le fabriquant et également le mandant de ce projet. Ce système permet de répondre aux besoins des thérapeutes et d'offrir des thérapies variées en un seul dispositif paramétrable. Il consiste en un robot, une interface homme-machine (IHM) ainsi qu'une couche software.
Le robot est motorisé et permet de mobiliser les jambes dans le plan sagittal mais il possède également des capteurs pour détecter la force appliquée par le patient. Grâce à la couche software exécutée dans un environnement temps réel ainsi qu'à de nombreuses modélisations de mouvements, le système est capable de détecter l'intention de mouvement du patient et de l'accompagner dans une trajectoire définie. Finalement, l'interface homme machine permet au personnel médical d'entrer les données du patient ainsi que de lui choisir et paramétrer des séances de réhabilitation composées d'exercices paramétrables.
\\

Pour ce projet, ce robot est utilisé et considéré comme périphérique principal pour le SG (ce qui était déjà le cas dans le projet antérieur). Le SG peut avoir accès aux informations physiques du patient ainsi que les valeurs des capteurs ou d'autres données traitées par la couche software (tel que la position des pieds dans un repère donné). Du point de vue du thérapeute ou de l'IHM, le SG doit être considéré comme un nouvel exercice.
\\

D'autres facteurs tels que les neurones miroirs peuvent également influer sur la qualité de la réhabilitation. Ces dernières peuvent être sollicitées notamment par le placement du patient dans une réalité virtuelle où une représentation du mouvement est présente. Cet effet est renforcé avec l'immersion du patient. Trois sens sont visés pour l'immersion. La vue: à l'aide d'un casque de réalité virtuelle; L'ouïe: à l'aide d'un casque sonore, de sons 3D et d'une musique d'ambiance; Le toucher: via le retour haptique du robot.
\\

D'autres périphériques additionnels sont envisageables, tels que, par exemple, une manette de jeu, une souris, un clavier, etc. Ces derniers doivent cependant éviter de détériorer la qualité de l'interaction avec le robot. Les interactions additionnelles peuvent offrir un \textit{gameplay} bien plus intéressant avec un challenge pouvant être utile en réhabilitation qui est de détourner l'attention du mouvement principal. Cependant, de nombreux patients peuvent, dû à leur traumatisme du système nerveux, avoir d'énormes difficultés à effectuer ces actions simultanément. Certains de ces patients auront même beaucoup de peine à traiter une scène riche en informations, contenant beaucoup de détails et d'éléments. Il faut donc prévoir un moyen d'appauvrir une scène voire même d'y placer quelques objets familiers, tel qu'un cadre ou un meuble, afin de rassurer le patient.
\\
%TODO Parle des SG et 2-3 études )Voir rapport NZA)

Les objectifs de ce projet sont donc les suivants. Établir une base documentaire et un état de l'art basés sur les différentes caractéristiques du projet R\&D englobant. En se basant sur le point précédent, concevoir le SG dans sa finalité. Puis, sélectionner et prioriser les objectifs du projet englobant pouvant être réalisés dans ce travail. Finalement, implémenter les objectifs sélectionnés.
\\

Ce projet suit des méthodologies Agile. La réalisation de prototypes pour la validation de fonctionnalités est préférée à une définition complète et précise de l'ensemble du SG. La conception ne décrit que globalement le but du jeu et ses principales caractéristiques. De plus, suite à la démonstration de certains prototypes, la priorité de certains objectifs peut être réévaluée durant le développement.
\\

Ce rapport résume l'ensemble du travail effectué. Il commence par un état de l'art de la réalité virtuelle et des SGs dans l'évaluation et la réhabilitation de troubles cognitifs et de mouvements. Il est suivit d'une analyse de la situation actuelle et des éléments à prendre en compte pour la suite. Le chapitre suivant -- "Conception" -- détaille les caractéristiques du SG pouvant être implémenté durant le projet R\&D et en sélectionne pour ce travail. Le chapitre suivant, "Développement", détaille la réalisation des principales fonctionnalités. Ensuite, le chapitre "Résultats" résume le SG actuel et présente des mesures et retours récoltés. Finalement, une liste des perspectives et une conclusion closent ce document.