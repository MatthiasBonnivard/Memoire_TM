\begin{itemize}
	\item \textbf{Avatar --} Représentation du joueur dans un environnement virtuel. Dans le SG développé, l'avatar est humanoïde et possède une combinaison spatiale;
	
	\item \textbf{API --} De l'anglais, "\textit{Application Programming Interface}", définit une bibliothèque logicielle contenant un ensemble de classes agissant comme interface de programmation;
	
	\item \textbf{AVC --} Désigne un "Accident Vasculaire Cérébral", parfois appelé "Attaque Cérébrale". Un infarctus ou une hémorragie au niveau du cerveau qui cause un déficit neurologique. Cet accident peut induire une hémiplégie et/ou une héminégligence. Une partie du public cible du SG sont des personnes ayant été victime d'un AVC;
	
	\item \textbf{Billboard --} Un \textit{billboard} est, en imagerie numérique, un quadrilatère sur lequel on applique une texture. Ils sont souvent utilisés pour simplifier des objets complexes tels que des arbres ou encore des tirs ou des explosions. Afin d'éviter que l'utilisateur ne voit qu'ils n'ont aucune épaisseur, ils s'orientent toujours face à la caméra;
	
	\item \textbf{Billboards, ensemble de --} Aussi appelé "Billboard cloud", il représente la combinaison de plusieurs \textit{billboard} affichant des textures différentes et ne s'orientant pas face à la caméra. Il permet de réduire drastiquement la complexité d'un modèle 3D tout en le représentant de façon distincte \cite{Behrendt_BillboardCloud, Decoret_BillboardCloud, Decoret_BillboardCloudExtrem}. Dans ce projet, les ensembles de \textit{billboards} créés n'utilisent, au maximum, que deux plans verticaux se coupant perpendiculairement en leurs centre horizontaux. Ils sont une des géométries permettant l'appauvrissement géométrique de certains objets de la scène;
	
	\item \textbf{Cycling --} Exercice/mouvement de pédalage circulaire déjà implémenté et utilisable dans le système LAMBDA. Inspiré de thérapies existantes. Souvent asymétrique si non-précisé;
	
	\item \textbf{Elypsoidal cycling --} Identique au cycling, cependant le mouvement n'est pas circulaire mais suis une ellipse paramétrable;
	
	\item \textbf{Framework --} Ensemble de composants logiciels permettant la création d'une partie ou de l'intégralité d'un logiciel. Se distingue d'une API par un caractère plus générique et des contraintes de constructions guidant l'architecture du logiciel;
	
	\item \textbf{Game flow --} Le \textit{game flow} est le ressentit de l'utilisateur sur le challenge que le jeu demande, le sentiment d'avoir le contrôle sur le jeu \cite{Salen_RulesOfPlay}. Il est bon si l'équilibre entre le challenge demandé et les capacités de la personne est respecté. Trop de challenge induirait de l'anxiété et trop peu de l'ennui. Le niveau optimal se situe lorsque la tâche est difficile mais faisable \cite{Green_ExercisingBrain}. C'est dans cet équilibre que le joueur aura une expérience optimale du jeu;
	
	\item \textbf{Gameplay --} Terme caractérisant le ressenti du joueur quand il utilise un jeu vidéo;
	
	\item \textbf{Hémiparésie --} Désigne une parésie d'un seul coté du corps (droit ou gauche);
	
	\item \textbf{Héminégligence --} Déficit cognitif pouvant être une séquelle d'une lésion cérébrale. Aussi appelée "négligence spatiale unilatérale", elle désigne une difficulté à porter son attention, à explorer ou à réaliser des mouvements du côté opposé à la lésion \cite{InformationHeminegligence}; 
	
	\item \textbf{HMD --} De l'anglais, "\textit{Head-Mounted Display}". Désigne un casque de réalité virtuelle, aussi appelé "visiocasque". Il est porté sur la tête et possède un ou deux écrans permettant d'afficher du contenu devant un seul ou les deux yeux. Certains permettent un affichage stéréoscopique pouvant recréer un effet de profondeur. D'autres possèdes des capteurs permettant de savoir l'orientation de la tête de l'utilisateur et pouvant être utilisés pour changer le champ de vision dans la réalité virtuelle;
	
	\item \textbf{HUD --} De l'anglais, "\textit{Head-Up Display}". Désigne un affichage pouvant être consulté en gardant la tête droite (à l'inverse d'un tableau de bord). Souvent utilisé dans les jeux vidéo pour afficher des informations sur le personnage ou encore un aperçu de la carte du monde;
	
	\item \textbf{IHM --} Interface Home Machine. Permet au thérapeute de gérer le profil de ses patients et leurs exercices;
	
	\item \textbf{I/O --} Entrées-sorties. Acronyme, de l'anglais "\textit{Input/Output}". Terme englobant tous les échanges d'un périphérique, d'un composant ou d'un programme informatique. Dans ce rapport, peut désigner les échanges du SG avec l'utilisateur ou avec le LHS;
	
	\item \textbf{Kinect --} Périphérique initialement destiné à une console de jeu afin de se passer de manettes. À l'aide de lumière structurée, ce périphérique composé de plusieurs capteurs peut détecter des mouvements et la structure d'une personne placée en face \cite{Kinect_website};
	
	\item \textbf{Latence --} Désigne le délai de transmission d'une information, de la source au destinataire. Dans ce travail, la latence peut définir une latence visuelle induite par les calculs de la carte graphique ou une latence de récupération des données du LHS;
	
	\item \textbf{Lambda --} Produit de \textit{Lambda Health System} S.A., composé d'un robot, d'une couche software exécutée en temps réel et d'une IHM. Peut être utilisé pour désigner uniquement le robot;
	
	\item \textbf{LHS (S.A.) --} Mandant du projet. Start-up propriétaire du produit LAMBDA;
	
	\item \textbf{Mouvement asymétrique --} Les deux jambes réalisent le même mouvement décalé d'un demi-cycle;
	
	\item \textbf{Mouvement symétrique --} Les deux jambes réalisent la même action en même temps;
	
	\item \textbf{MSE --} Acronyme, de l'anglais "\textit{Master of Science in Engineering}", de la formation dont ce projet est le travail final;
	
	\item \textbf{MVC --} Acronyme "Modèle Vue Contrôleur" ou, en anglais, "\textit{Model View Controller}". Définit un patron d'architecture logicielle visant à séparer la logique du code en trois parties et ainsi mieux organiser les sources;
	
	\item \textbf{Neurones miroirs --}  Les neurones miroirs sont des neurones qui s'activent à la fois lors de l'exécution d'un mouvement et durant l'observation de ce même mouvement effectué par une autre personne. Elles permettent de favoriser l'apprentissage du mouvement en question;
	
	\item \textbf{Ocf --} API de communication crée par Objectis \cite{OcfClient_website} et utilisée par le LHS. C'est grâce à cette API que le SG peut communiquer avec le LHS;
	
	\item \textbf{Oculus Rift DK2 --} Kit de développement deux de l'\textit{Oculus Rift} \cite{OculusDk2_website}. HMD utilisé durant ce projet et les précédents;
	
	\item \textbf{Parésie --} Une parésie est une perte partielle des capacités motrices. Elle se distingue d'une paralysie (ou plégie) qui est une perte totale de la motricité;
	
	\item \textbf{Parétique --} Se dit d'un membre souffrant de parésie;
	
	\item \textbf{Plan sagittal --} Plan de coupe traversant une personne verticalement, comme le ferait une flèche tirée par un observateur en face d'elle;
	
	\item \textbf{Press-leg --} Exercice/mouvement déjà implémenté et utilisable dans le système Lambda. Consiste à tendre les jambes puis à les replier. Un poids fictif peut être créé afin de devoir le pousser et le retenir. Inspiré de thérapies existantes;
	
	\item \textbf{SG --} De l'anglais "\textit{Serious Game}", jeu avec une intention sérieuse (dans ce projet: la réhabilitation);
	
	\item \textbf{Shader --} Programme utilisé dans la carte graphique pour influer sur une ou plusieurs étapes du rendu qu'elle effectue;
	
	\item \textbf{Skybox --} Technique d'infographie permettant de donner l'illusion d'un décor, d'un espace étendu. C'est un décor projeté sur les six faces intérieur d'un cube depuis son centre. La caméra est alors placée en son centre à distance constante, le rendant alors inatteignable et toujours projeté correctement pour l'utilisateur (ne voyant pas un cube, mais un paysage complet);
	
	\item \textbf{VE --} Acronyme, de l'anglais "\textit{Virtual Environment}", désigne un environnement virtuel pouvant être affiché sur un écran ou utilisé dans une réalité virtuelle;
	
	\item \textbf{VR --} Acronyme, de l'anglais "\textit{Virtual Reality}", désigne une réalité virtuelle. L'utilisation d'un environnement virtuel avec une présence simulée à l'aide de stimuli sensoriels.
	
\end{itemize}

