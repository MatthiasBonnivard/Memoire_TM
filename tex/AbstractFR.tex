Ce travail a pour but la création d'un \textit{serious game} (SG) pour la neuroréhabilitation des membres inférieurs utilisant le périphérique de réhabilitation \textit{Lambda Health System} (LHS). Les utilisateurs sont des personnes souffrant de parésies de ces mêmes membres, ils doivent réapprendre un mouvement de marche. Ces personnes souffrent également fréquemment de troubles cognitifs, dont notamment l'héminégligence, à prendre en compte pour le SG.
\\

Le but principal du SG est donc d'ajouter un aspect ludique à un exercice de physiothérapie consistant à la répétition d'un mouvement de marche. Ce SG a également trois autres aspects importants: une haute immersion; un environnement pouvant être appauvrit pour s'adapter aux troubles cognitifs du patient; une visualisation du mouvement de marche. Ce dernier a pour but d'activer les neurones miroirs et ainsi d'améliorer le processus de réhabilitation.
\\
 
Le SG implémenté consiste en l'exploration de planètes à la recherche de traces d'une civilisation disparue. Ceci sur des parcours guidés où le patient doit avancer en marchant. La répartition gauche/droite des vestiges le long du chemin est définie par le thérapeute.
\\

La recherche de ces traces se fait à l'aide d'un casque de réalité virtuelle. Ce périphérique est connu pour être un des meilleurs actuel pour une immersion visuelle. Le sens de l'ouïe est stimulé à l'aide de bruits de pas, de sons d'ambiance, de musiques et de sons d'interface. Le toucher est également visé à l'aide du retour haptique du LHS. Il n'est cependant pas présent actuellement. Pour l'appauvrissement, neuf facteurs ont été réalisés, influant sur la géométrie, les effets de lumière, les animations, les sons, les couleurs et la densité des éléments. Concernant la représentation du mouvement, le joueur joue à la première personne et les jambes de son avatar sont articulées pour correspondre aux mouvements des siennes. Un autre personnage virtuel est placé devant le joueur durant tout le parcours et possède une animation de marche.
\\

Dû à des tests officiels réalisés au Centre Hospitalier Universitaire Vaudois (CHUV), l'intégration au LHS, bien que préparée, n'a pas pu être testée. Le mouvement est, par conséquent, simulé au clavier. Une rencontre avec un médecin a cependant permis d'obtenir des retours encourageants sur le SG actuel, même en dehors d'une réhabilitation de marche pour des personnes souffrant de troubles cognitifs.
\\

Mots clés: \textit{serious game}, neuroréhabilitation des membres inférieurs, réalité virtuelle, monde virtuel, immersion, appauvrissement cognitif, héminégligence, multimodal.
