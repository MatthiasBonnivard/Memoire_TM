Tout d'abord, je tiens à remercier le Prof. Dr. Stéphane Gobron pour ses conseils et son encadrement tout au long de ce travail.
Je remercie Messieurs Aurélien Fauquex et Yannick Charrotton de "Lambda Health System S.A." pour les différentes séances réalisées durant ce projet ainsi que pour le travail fournit durant le projet précédent, notamment au niveau de la communication des SGs au LHS qui a été réutilisée.
Concernant également la communication mais aussi l'intégration du HMD, je tiens à remercier Monsieur Nicolas Zannini pour avoir effectué ces tâches durant son travail de Master ainsi que pour sa disponibilité durant ce projet.
\\

Je tiens également à remercier Monsieur Kevin Laipe ayant rejoint le groupe de compétences imagerie de la HE-Arc en tant que stagiaire durant ce travail. Son aide s'est avérée précieuse pour la création de niveaux supplémentaires et également pour la création de nouvelles fonctionnalités. Je le remercie aussi pour le temps passé à tester les fonctionnalités du SG, ce qui a permis d'identifier à temps certains \textit{bugs} et de les corriger.
\\

Je tiens à remercier le Prof. Dr. Michel Lauria de la HES-SO de Genève d'avoir participé à la définition et à la sélection des objectifs ainsi qu'à l'élaboration du concept du SG.
Je tiens à remercier le Dr. Rolf Frischnecht, médecin cadre du CHUV et membre de l'équipe LHS pour s'être déplacé à St-Imier et avoir testé le SG. Les retours recueillis, issus des nombreuses minutes passées en immersion dans la réalité virtuelle, sont encourageants et permettent d'envisager la suite de ce travail d'après un avis du domaine médical.
\\

Je remercie également Madame Améthyste Molin pour la relecture de ce document et les corrections pertinentes apportées.
\\

Je remercie d’avance Monsieur Quentin Silvestre, l’expert de mon travail de Master, pour le temps qu’il consacrera à la lecture de ce mémoire et pour sa future participation à ma défense de Master.